\documentclass{article}
\usepackage{cite}
\usepackage{amsmath}
\usepackage{bm}
\usepackage{booktabs}
\usepackage{cool}

\DeclareMathOperator{\divr}{div}
\DeclareMathOperator{\grad}{grad}

\numberwithin{equation}{section}

\newtheorem{defn}{Definition}
\newtheorem{thm}{Theorem}



\title{The existence of a strong solution to the Navier-Stokes equations}
\author{Mukhtarbay Otelbaev\\
Institute of Mathematics and Mathematical Modeling MES
RK\thanks{Ministry of Education and Science, Republic of Kazakhstan}\\
125 Pushkin St, Almaty 050010, email: otelbaevm@mail.ru}


\date{ISSN 1682-0525. Mathematical Journal. 2013. Vol 13, Num 4 (50).\\
Translated by Mikhail Wolfson, Ph.D., \today}

\begin{document}
\maketitle

\begin{abstract}
    In this work, a solution to the sixth Millennium Prize Problem is provided:
    the existence and uniqueness of a strong solution to the three-dimensional
    Navier-Stokes problem with periodic spatial boundary conditions.

    Key words: sixth Millennium Prize problem, Navier-Stokes equation, strong
    solution.
\end{abstract}

\section{Introduction}

\subsection{A short history of the problem}

The problem of describing the dynamics of an incompressible fluid, due to its
significance for both theory and application, has attracted the attention of
many researchers. During the middle part of 2000, this problem was formulated
as the \emph{sixth Millennium Prize problem}, about the existence and
smoothness of a solution to the Navier-Stokes equations for an incompressible
viscous fluid \cite{clay}.

The solution to this problem was the subject of many works even before its
announcement as a Millennium Prize problem. As the number of these works is
too large, I will not include a list. Deep results, in my opinion, were
arrived at in the works of O. A. Ladyzhenskaya \cite{lad1,lad2,lad3,lad4} and R.
Temam \cite{tem1, tem2}. This problem has been of interest to many first-class
mathematicians, who have been able to solve important mathematical problems,
including problems in hydrodynamics. Substantial results have been arrived at
in the works of such great mathematicians a A. N. Kolmogorov \cite{kol1}, 
J. Leray \cite{ler1, ler2}, E. Hopf \cite{hop1}, J.-L. Lions \cite{lio1,
lio2}, M. I. Vishik \cite{vis1}, V. A. Solonnikov \cite{sol1}, and many
others. Of course, this list is far from complete. It is not my goal to provide
a full overview of existing works. With the exception of those publications,
whose results I have directly used in the writing of this work (cf.
\cite{ote1,ote2,ote3,ote4,ote5,ote6,ote7,ote8,ote9,ote10,ote11,ote12,ote13,ote14,ote15,ote16,ote17,ote18,ote19,ote20,ote21,ote22,sol2,aky1,jak1,aby1,aby2,aby3,aby4,aby5,ait1}).
My own works are included in this list, sometimes with coauthors, and so are
the works of many other Kazakh mathematicians, whose positive influence I have
constantly benefited from.

A full solution to the two-dimensional problem is provided by O. A.
Ladyzhenskaya in \cite{lad1}. In \cite{lad4}, she provides a sufficiently full
analysis of the current state of the problem and an overview of existing
literature and proposed solution methods. In part, the main problem of a
global unique solution to the three-dimensional Navier-Stokes is cast as the
problem of finding a special \emph{a priori} bound for all possible solutions.

One must also keep in mind the existence of a large number of works containing
mistakes or lacking proofs that have been published in little-known journals
or merited electronic publication. Despite their  mistakes, these works
deserve respect.

I have been studying this problem since 1980. All of my works, written
either by myself or with coauthors, published after 1982 and devoted to
nonlinear equations, approximation techniques for solving equations and the
final inverse problem, used ideas and technical approaches developed during my
(and also some of my students' and coauthors') unsuccessful attempts to solve
the problem of the strong solvability of the Navier-Stokes equation (cf.\ for
example, 
\cite{ote1,ote2,ote3,ote4,ote5,ote6,ote7,ote8,ote9,ote10,ote11,ote12,ote13,ote14,ote15,ote16,ote17,ote18,ote19,ote20,ote21,ote22}).

I would like to take this opportunity to express my gratitude to Professor M.
Sadybekov for his careful reading of the work. He also reworked 
Section 4, which I had briefly outlined, and completely wrote Section 8. His
comments proved very useful in the formulation of the final state of this
article.

I dedicate this work to the family of my dear teachers: the mathematics
professors T. I. Amanov, M. G. Gasymov, A. G. Kostuchenko, B. M. Levitan, P.
I. Lizorkin, and also my school teachers I. Adykeev and A. M. Panivanov.

\subsection{Statement of the problem}
Let $Q \subset R^3$ be a domain in three dimensions, $\Omega = (0,a)\times Q,
a > 0$. In this work we examine the case when $Q$ is a three-dimensional cube
centered at the origin and edges of length $2\pi$, parallel to the coordinate axes.

The \emph{Navier-Stokes problem} consists of finding the unknowns:
\begin{align*}
&\text{the velocity vector } u(t,x) = (u_1(t,x), u_2(t,x), u_3(t,x)),\\
&\text{and the pressure---a scalar function } p(t,x)
\end{align*}
for points $x \in Q$ at time $t \in (0, a)$ that satisfy the system of
equations
\begin{equation}\label{ns}
    \left\{
    \begin{aligned}
        &\pderiv{u_j}{t} + \sum_{k=1}^{3} u_k \pderiv{u_j}{x_k} = \Delta u_j -
        \pderiv{p}{x_j} + f_j, \quad (t,x) \in \Omega, \qquad j = 1,2,3;\\
       &\divr u \equiv \sum_{k=1}^{3} = 0,\qquad (t,x) \in \Omega.
    \end{aligned}
    \right.
\end{equation}

Here, $f(t,x) = (f_1(t,x), f_2(t,x), f_3(t,x))$ is an external force, and 
$\Delta = \sum_{k=1}^{3} \pderiv[2]{}{x}$ is the Laplacian for spatial
dimensions, and the coefficient of viscosity, $\nu$, is taken to be 1, without
loss of generality.

To the system of equations in \eqref{ns}, we add initial and final time
boundary conditions (along the spatial dimensions, we invoke periodic boundary
conditions).  Without loss of generality, we can take the initial conditions
to be zero:

\begin{equation}
    u(t,x)\rvert_{t=0}, \quad x\in \bar{Q};
    \label{nsibc}
\end{equation}
\begin{align}
    u(t,x)\rvert_{x_k=-\pi} &= u(t,x)\rvert_{x_k = \pi},\nonumber\\
    p(t,x)\rvert_{x_k=-\pi} &= p(t,x)\rvert_{x_k = \pi},
    &k=1,2,3; \quad 0\leq t \leq a. \label{nspbc} \\
    \left.\pderiv{u}{x_k}\right\rvert_{x_k=-\pi} &=
    \left.\pderiv{u}{x_k}\right\rvert_{x_k = \pi},\nonumber
\end{align}

The system of equations \eqref{ns} and initial/boundary constraints
\eqref{nsibc}, \eqref{nspbc} do not allow a unique solution to the pressure
$p(t,x)$. For this reason, we add the constraint
\begin{equation}
    \int_Q p(t,x)\,dx = p_0,\ p_0 = \text{const} > 0.
    \label{nspc}
\end{equation}

In the problem, the sought-after quantities are the vector velocity function $u
= (u_1, u_2, u_3)$ and the scalar pressure function $p$. We will denote \emph{solution to
the problem} as the pair $(u;p) = (u_1, u_2, u_3; p)$.

We could have examined the case where, instead of $Q$, a more general domain
is used, and instead the period boundary conditions in \eqref{nspbc}, other
conditions are used (for example, ``sticky'' boundary conditions). We do not
wish to complicate this work with such extensions only for technical reasons.
All of our basic innovations and analytical technique will be demonstrated in
this work for the case of periodic boundary conditions.

Furthermore, I intend to write at least one other work devoted to the case of
general initial/boundary problems for a system of hydrodynamic equations.

\subsection{Essential symbols and definitions}

With $L_2(\Omega)$, as usual, we denote the Hilbert space of Lebesgue vector
functions $f(t,x) = (f_1(t,x), f_2(t,x), f_3(t,x)) \in R^3$ with the scalar product
\[
    (f,g) = \int_{\Omega} \langle f(t,x), g(t,x)\rangle\, dx dt \equiv
    \int_0^a\left( \int_Q\langle f(t,x), g(t,x)\rangle\,dx \right)\, dt
\]
and the norm $\lVert f \rVert = \sqrt{ (f, f) }$. Here and throughout,
$\langle f, g \rangle$ is the scalar product of the vectors $f$ and $g$ in the
Euclidian space $R^3$.

For the sake of the conciseness of our notation, we use the standard symbols:
\begin{gather*}
    \nabla = \left(\pderiv{}{x_1}, \pderiv{p}{x_2}, \pderiv{p}{x_3}\right);\ 
    \langle u, \nabla u \rangle = \left(
        \sum_{j=1}^3u_j\pderiv{u_1}{x_j},
        \sum_{j=1}^3u_j\pderiv{u_2}{x_j},
        \sum_{j=1}^3u_j\pderiv{u_3}{x_j},
    \right);\\
    \grad p \equiv \nabla p = \left(\pderiv{p}{x_1}, \pderiv{p}{x_2},
    \pderiv{p}{x_3}\right).
\end{gather*}

\begin{defn}
    Given $f\in L_2(\Omega)$, we will call the solution $(u;p) = (u_1, u_2, u_3; p)$
    of the problem \eqref{ns} -- \eqref{nspc} \textbf{strong} if
    \[
        \pderiv u t, \Delta u, \langle u, \nabla \rangle u, \grad p \in L_2(\Omega).
    \]
\end{defn}

\begin{defn}
    We will say that the Navier-Stokes problem is \textbf{strongly solvable}
    in $L_2(\Omega)$, if for any $f\in L_2(\Omega)$, the problem \eqref{ns} -- \eqref{nspc}
    has a single strong solution.
\end{defn}

\section{Statement of the main result}

E. Hopf's classic result \cite{hop1}, which proved that the problem \eqref{ns} -- \eqref{nspc}
has a generalized solution satisfying the bound
\begin{equation}
    \sum_{k=1}^3\left( \lVert u_k(t,\cdot) \rVert^{2}_{L_2(Q)} +
    \int_0^t \lVert \grad u_k(\eta, \cdot)\rVert ^2 _{L_2(Q)}\,d\eta \right)
    \leq \sum_{k=1}^3\int_0^t \lVert f_k(\eta, \cdot)\rVert ^2 _{L_2(Q)}\,d\eta
    \label{hopest}
\end{equation}
is well known.

This estimate is arrived at easily if the $k$-th equation of the system is
multiplied by $u_k$, and, noting the equality $\divr u = 0$, as well as the
initial \eqref{nsibc} and boundary $\eqref{nspbc}$ conditions, the equations are
all integrated and summed for all $k=1,2,3$.

In the case when the number of spatial variables is not less than three, the
bound \eqref{hopest} is insufficient for perturbation theory. In my option,
this fact in particular is one of the most important reasons why the problem
of the strong solvability of the Navier-Stokes equations is a Millenium Prize
problem.

The main result of this paper is

\begin{thm}
    Given any $f\in L_2(\Omega)$, the problem \eqref{ns} -- \eqref{nspc} has a
    unique strong solution $(u;p)$ and this solution satisfies the bound
    \begin{equation}
        \left\lVert \pderiv{u}{t} \right\rVert + \lVert \Delta u \rVert +
        \lVert (u, \nabla)u \rVert + \lVert\grad p\rVert \leq 
        C \left( 1 + \lVert f \rVert + \lVert f \rVert^l \right)
        \label{soln}
    \end{equation}
    where $\lVert \cdot \rVert$ is the norm on $L_2(\Omega)$, and the
    constants $C > 0$ and $l \geq 1$ do not depend on $f \in L_2(\Omega)$.
\end{thm}

This theorem presents a complete solution to the \emph{sixth Millenium Prize
problem} about the existence and smoothness of solutions to the Navier-Stokes
equations for an incompressible viscous fluid \cite{clay}. This result also
enables the use of perturbation theory, and improving the smoothness of a
solution with increasing smoothness in the problem data.

We note that if the spatial dimension is greater than five, the strong
solvability (in the sense of our definitions) would prevent the use of
perturbation theory.

In this work I have concerned myself only with the three-dimensional case.
Therefore, I have chosen definitions convenient for me. The case when the
spatial dimension is greater than 3 will be examined in another work, which
will also include certain cases of general boundary conditions.

\begin{thebibliography}{99}
    \bibitem{clay} Fefferman Ch. Existence and Smoothness of the Navier-Stokes
        equation.
    \bibitem{lad1} Ladyzhenskaya 1.
    \bibitem{lad2} Ladyzhenskaya 2.
    \bibitem{lad3} Ladyzhenskaya 3.
    \bibitem{lad4} Ladyzhenskaya 4.
    \bibitem{tem1} Temam 1.
    \bibitem{tem2} Temam 2.
    \bibitem{kol1} Kolmogorov 1.
    \bibitem{ler1} Leray 1.
    \bibitem{ler2} Leray 2.
    \bibitem{hop1} Hopf 1.
    \bibitem{lio1} Lions 1.
    \bibitem{lio2} Lions 2.
    \bibitem{vis1} Vishik 1.
    \bibitem{sol1} Solonnikov 1.
    \bibitem{ote1} Otelbaev 1.
    \bibitem{ote2} Otelbaev 2.
    \bibitem{ote3} Otelbaev 3.
    \bibitem{ote4} Otelbaev 4.
    \bibitem{ote5} Otelbaev 5.
    \bibitem{ote6} Otelbaev 6.
    \bibitem{ote7} Otelbaev 7.
    \bibitem{ote8} Otelbaev 8.
    \bibitem{ote9} Otelbaev 9.
    \bibitem{ote10} Otelbaev 10.
    \bibitem{ote11} Otelbaev 11.
    \bibitem{ote12} Otelbaev 12.
    \bibitem{ote13} Otelbaev 13.
    \bibitem{ote14} Otelbaev 14.
    \bibitem{ote15} Otelbaev 15.
    \bibitem{ote16} Otelbaev 16.
    \bibitem{ote17} Otelbaev 17.
    \bibitem{ote18} Otelbaev 18.
    \bibitem{ote19} Otelbaev 19.
    \bibitem{ote20} Otelbaev 20.
    \bibitem{ote21} Otelbaev 21.
    \bibitem{ote22} Otelbaev 22.
    \bibitem{sol2} Solonnikov 2.
    \bibitem{aky1} Akysh 2.
    \bibitem{jak1} Jakupov 1.
    \bibitem{aby1} Abylkairov 1.
    \bibitem{aby2} Abylkairov 2.
    \bibitem{aby3} Abylkairov 3.
    \bibitem{aby4} Abylkairov 4.
    \bibitem{aby5} Abylkairov 5.
    \bibitem{ait1} Aitzhanov 1.
    \bibitem{bai1} Baitulenov 1.
    \bibitem{sol3} Solonnikov 3.
    \bibitem{sak1} Saks 1.
    \bibitem{sch1} Schmeisser 1.
    \bibitem{ote23} Otelbaev 23.
\end{thebibliography}


%\appendix
%\section{Supplementary Information}

\end{document}
